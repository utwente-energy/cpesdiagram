% Copyright 2025 University of Twente
%
% Licensed under the Apache License, Version 2.0 (the "License");
% you may not use this file except in compliance with the License.
% You may obtain a copy of the License at
%
%     http://www.apache.org/licenses/LICENSE-2.0
%
% Unless required by applicable law or agreed to in writing, software
% distributed under the License is distributed on an "AS IS" BASIS,
% WITHOUT WARRANTIES OR CONDITIONS OF ANY KIND, either express or implied.
% See the License for the specific language governing permissions and
% limitations under the License.

%Tikz styles
\def \importcpesnodestyle{
\tikzset{
% Normal node styles
  % Devices
  device/.style={rectangle, rounded corners=2pt, inner sep=12pt, draw=blue, top color= white, bottom color=blue!25, text=black},
  % Physical
  phy/.style={circle, draw=black, top color= white, inner sep=8pt, bottom color=gray!25, text=black},
  % Cyber controllers
  control/.style={regular polygon, regular polygon sides=6, rounded corners=1pt, draw=teal, top color=white, inner sep=8pt, bottom color=teal!25, text=black},
  % Societal interaction
  interact/.style={diamond, rounded corners=2pt, inner sep=8pt, draw=red, top color= white, bottom color=red!25, text=black},
  %
  % Node connector style (small icons to connect lines)
  % Devices
  dnode/.style={rectangle, inner sep=3pt, fill=blue, draw=blue}, text=black,
  % Physical flow
  pnode/.style={circle,draw=black, inner sep=2pt, fill=black, text=black},
  % Cyber controllers
  cnode/.style={regular polygon,regular polygon sides=6,draw=teal, inner sep=2pt, fill=teal, text=black},
  % Societal interaction
  inode/.style={diamond, draw=red, inner sep=1.5pt, fill=red, text=black}
  }
}

\def \importcpeslinestyle{
\tikzset{
% Style for links between nodes
  % Devices
  dlink/.style={rounded corners=5pt, blue, text=black},
  % Physical
  plink/.style={thick, rounded corners=5pt, text=black},
  % Cyber controllers
  clink/.style={thick, teal, rounded corners=5pt, text=black},
  % Cyber controllers (uni-directional)
  clinku/.style={densely dotted, thick, teal, rounded corners=5pt, text=black},
  % Societal interaction
  ilink/.style={red, rounded corners=5pt, text=black}
  }
}


\def \importcpesnodestylegrey{
\tikzset{
% Normal node styles
  % Devices
  device/.style={rectangle, rounded corners=2pt, inner sep=12pt, draw=black, top color=white, bottom color=gray!25, text=black},
  % Physical
  phy/.style={circle, draw=black, top color= white, inner sep=8pt, bottom color=gray!25, text=black},
  % Cyber controllers
  control/.style={regular polygon,regular polygon sides=6, rounded corners=1pt, draw=gray, top color=white, inner sep=8pt, bottom color=gray!25, text=black},
  % Societal interaction
  interact/.style={diamond, rounded corners=2pt, inner sep=8pt, draw=gray, top color=white, bottom color=gray!25, text=black},
  %
  % Node connector style (small icons to connect lines)
  % Devices
  dnode/.style={rectangle, inner sep=3pt, fill=black, text=black},
  % Physical
  pnode/.style={circle,draw=black, inner sep=2pt, fill=black, text=black},
  % Cyber controllers
  cnode/.style={regular polygon,regular polygon sides=6,draw=gray, inner sep=2pt, fill=gray, text=black},
  % Societal interaction
  inode/.style={diamond, draw=gray, inner sep=1.5pt, fill=gray, text=black}
  }
}

\def \importcpeslinestylegrey{
\tikzset{
% Style for links between nodes
  % Devices
  dlink/.style={rounded corners=5pt, black, text=black},
  % Physical flow
  plink/.style={thick, rounded corners=5pt, text=black},
  % Cyber controllers
  clink/.style={thick, gray, rounded corners=5pt, text=black},
  % Cyber controllers (uni-directional)
  clinku/.style={densely dotted, thick, gray, rounded corners=5pt, text=black},
  % Societal interaction
  ilink/.style={gray, rounded corners=5pt, text=black}
  }
}


% Import command for the coloured style
\def \importcpesstyle{
  \importcpesnodestyle
  \importcpeslinestyle
}


% Import command to use the greyscale version
\def \importcpesstylegrey{
  \importcpesnodestylegrey
  \importcpeslinestylegrey
}


% Import command to use the grayscale version instead if you prefer to use American English ;-)
\def \importcpesstylegray{
  \importcpesstylegrey
}


% Node definitions for placing in a matrix


% The large nodes
% parameters:
%   #1: unique name for identification
%   #2: icon name (as provided by the material design icons, see https://fonts.google.com/icons)
%   #3: label text with styling (e.g., label=right:\scriptsize{some device})
% for respectively device, physcial, controllers (cyber), and societal interaction
\def \devicenode#1#2#3{
  \node (#1) [device, #3] {\pgfbox[center,center]{\includegraphics[scale=0.5]{cpesdiag/icons/plain/#2.pdf}}}
}

\def \phynode#1#2#3{
  \node (#1) [phy, #3] {\pgfbox[center,center]{\includegraphics[scale=0.5]{cpesdiag/icons/plain/#2.pdf}}}
}

\def \controlnode#1#2#3{
  \node (#1) [control, #3] {\pgfbox[center,center]{\includegraphics[scale=0.5]{cpesdiag/icons/plain/#2.pdf}}}
}

\def \interactionnode#1#2#3{
  \node (#1) [interact, #3] {\pgfbox[center,center]{\includegraphics[scale=0.5]{cpesdiag/icons/plain/#2.pdf}}}
}


% The small connection nodes
% parameters:
%   #1: unique name for identification
%   #2: label text with styling (e.g., label=right:\scriptsize{some device})
% for respectively device, physcial, controllers (cyber), and societal interaction
\def \dnode#1#2{
  \node (#1) [dnode, #2] {}
}

\def \pnode#1#2{
  \node (#1) [pnode, #2] {}
}

\def \cnode#1#2{
  \node (#1) [cnode, #2] {}
}

\def \inode#1#2{
  \node (#1) [inode, #2] {}
}




% Node definitions using coordinate system

% The large nodes
% parameters:
%   #1: unique name for identification
%   #2: coordinates (e.g., (1,1))
%   #3: icon name (as provided by the material design icons, see https://fonts.google.com/icons)
%   #4: label text with styling (e.g., label=right:\scriptsize{some device})
% for respectively device, physcial, controllers (cyber), and societal interaction
\def \devicenodex#1#2#3#4{
  \node (#1) at (#2) [device, #4] {\pgfbox[center,center]{\includegraphics[scale=0.5]{cpesdiag/icons/plain/#3.pdf}}}
}

\def \phynodex#1#2#3#4{
  \node (#1) at (#2) [phy, #4] {\pgfbox[center,center]{\includegraphics[scale=0.5]{cpesdiag/icons/plain/#3.pdf}}}
}

\def \controlnodex#1#2#3#4{
  \node (#1) at (#2) [control, #4] {\pgfbox[center,center]{\includegraphics[scale=0.5]{cpesdiag/icons/plain/#3.pdf}}}
}

\def \interactnodex#1#2#3#4{
  \node (#1) at (#2) [interact, #4] {\pgfbox[center,center]{\includegraphics[scale=0.5]{cpesdiag/icons/plain/#3.pdf}}}
}


% The large node swith dual icons
% parameters:
%   #1: unique name for identification
%   #2: coordinates (e.g., (1,1))
%   #3: large icon name (right, transparent) (as provided by the material design icons, see https://fonts.google.com/icons)
%   #4: small icon name (left bottom, filled version) (as provided by the material design icons, see https://fonts.google.com/icons)
%   #5: label text with styling (e.g., label=right:\scriptsize{some device})
% for respectively device, physcial, controllers (cyber), and societal interaction
\def \devicenodedx#1#2#3#4#5{
  \node (#1) at (#2) [device, #5] {};
  \pgfputat{\pgfxy(#2)}{\pgfbox[center,center]{\transparent{0.6}\includegraphics[scale=0.4, trim=0pt -10pt -10pt 0pt]{cpesdiag/icons/plain/#3.pdf}}};
  \pgfputat{\pgfxy(#2)}{\pgfbox[center,center]{\includegraphics[scale=0.25, trim=-28pt 0pt 0pt -28pt]{cpesdiag/icons/fill/#4.pdf}}}
}

\def \phynodedx#1#2#3#4#5{
  \node (#1) at (#2) [phy, #5] {};
  \pgfputat{\pgfxy(#2)}{\pgfbox[center,center]{\transparent{0.6}\includegraphics[scale=0.4, trim=0pt -10pt -10pt 0pt]{cpesdiag/icons/plain/#3.pdf}}};
  \pgfputat{\pgfxy(#2)}{\pgfbox[center,center]{\includegraphics[scale=0.25, trim=-28pt 0pt 0pt -28pt]{cpesdiag/icons/fill/#4.pdf}}}
}

\def \controlnodedx#1#2#3#4#5{
  \node (#1) at (#2) [control, #5] {};
  \pgfputat{\pgfxy(#2)}{\pgfbox[center,center]{\transparent{0.6}\includegraphics[scale=0.4, trim=0pt -10pt -10pt 0pt]{cpesdiag/icons/plain/#3.pdf}}};
  \pgfputat{\pgfxy(#2)}{\pgfbox[center,center]{\includegraphics[scale=0.25, trim=-28pt 0pt 0pt -28pt]{cpesdiag/icons/fill/#4.pdf}}}
}

\def \interactnodedx#1#2#3#4#5{
  \node (#1) at (#2) [interact, #5] {};
  \pgfputat{\pgfxy(#2)}{\pgfbox[center,center]{\transparent{0.6}\includegraphics[scale=0.4, trim=0pt -10pt -10pt 0pt]{cpesdiag/icons/plain/#3.pdf}}};
  \pgfputat{\pgfxy(#2)}{\pgfbox[center,center]{\includegraphics[scale=0.25, trim=-28pt 0pt 0pt -28pt]{cpesdiag/icons/fill/#4.pdf}}}
}

% The small connection nodes
%   #1: unique name for identification
%   #2: coordinates (e.g., (1,1))
%   #3: label text with styling (e.g., label=right:\scriptsize{some device})
% for respectively device, flow (physcial), controllers (cyber), and societal interaction
\def \dnodex#1#2#3{
  \node (#1) at (#2) [dnode, #3] {}
}

\def \pnodex#1#2#3{
  \node (#1) at (#2) [pnode, #3] {}
}

\def \cnodex#1#2#3{
  \node (#1) at (#2) [cnode, #3] {}
}

\def \inodex#1#2#3{
  \node (#1) at (#2) [cnode, #3] {}
}
